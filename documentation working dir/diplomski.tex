% Predlozak za pisanje diplomskog rada na PMF-MO
% Opcenita uputstva za LaTeX se mogu npr. naci na 
% http://web.math.hr/nastava/rp3, http://web.math.hr/nastava/s4-prof/latex.pdf
% NE PREPORUCA se "Ne baš tako kratak uvod u TEX", buduci se radi o vrlo starom prirucniku
% koji nije pogodan za moderne verzije LaTEXa.
% Originalna verzija "The not so short..." na http://tobi.oetiker.ch/lshort/lshort.pdf 
% je obnovljena i daje bolji uvid u moderne verzije LaTeXa

% Stil je optimiziran za kreiranje pdf dokumenta (npr. pomocu pdflatex-a, XeLaTeX-a)

\documentclass[a4paper,twoside,12pt]{memoir} % jednostrano: promijeniti twoside u oneside

% Paket inputenc omogucava direktno unosenje hrvatskih dijakritickih znakova 
% opcija utf8 za unicode (unix, linux, mac)
% opcija cp1250 za windowse
\usepackage[utf8]{inputenc}  % ukoliko se koristi XeLaTeX onda je \usepackage{xunicode}\usepackage{xltxtra}

% Stil za diplomski, unutra je ukljucena podrska za hrvatski jezik
\usepackage{diplomski}
% bibliografija na hrvatskom
\usepackage[languagenames,fixlanguage,croatian]{babelbib} % zahtijeva datoteku croatian.bdf
% hiperlinkovi 
\usepackage[pdftex]{hyperref} % ukoliko se koristi XeLaTeX onda je \usepackage[xetex]{hyperref}

% Odabir familije fontova:
% koristenjem XeLaTeX-a mogu se koristiti svi fontovi instalirani na racunalu, npr
% \defaultfontfeatures{Mapping=tex-text}
% \setmainfont[Ligatures={Common}]{Hoefler Text}
% ili
% \newcommand{\nas}[1]{\fontspec{Adobe Garamond Pro}\fontsize{24pt}{24pt}\color{Chocolate}\selectfont #1}
% i onda \nas{Naslov ...}
\usepackage{txfonts} % times new roman 

% Paket graphicx sluzi za manipuliranje grafikom 
\usepackage[pdftex]{graphicx} % ukoliko se koristi XeLaTeX onda je \usepackage[xetex]{graphicx}
% Paket amsmath je vec ukljucen
% Dodatno definirane matematicke okoline:
% teorem (okolina: thm), lema (okolina: lem), korolar (okolina: cor),
% propozicija (okolina: prop), definicija (okolina: defn), napomena (okolina: rem),
% slutnja (okolina: conj), primjer (okolina: exa), dokaz (okolina: proof)
% Definirane su naredbe za ispisivanje skupova N, Z, Q, R i C
% Definirane su naredbe za funkcije koje se u hrvatskoj notaciji oznacavaju drukcije 
% nego u americkoj: tg, ctg, ... (\tgh za tangens hiperbolni)
% Takodjer su definirane naredbe za Ker i Im (da bi se razlikovala od naredbe za imaginarni dio kompleksnog
% broja, naredba se zove \slika).

\pagestyle{headings}
% uz paket fancyhdr mogu se lako kreirati fancy zaglavlja i podnozja

% Podaci koje treba unijeti
\title{Paralelni algoritmi za problem grupiranja podataka}
\author{Anto Čabraja}
\advisor{prof. dr. sc. Goranka Nogo}  % obavezno s titulom (prof. dr. sc ili doc. dr. sc.)
\date{srpanj 2014.}  % oblika mjesec, godina

% Moguce je unijeti i posvetu
% Ukoliko nema posvete, dovoljno je iskomentirati/izbrisati sljedeci redak 
%\dedication{Samom sebi}

\begin{document}

% Naredna frontmatter generira naslovnu stranicu, stranicu za potpise povjerenstva, eventualnu posvetu i sadrzaj
% Moze se iskomentirati ukoliko nije u pitanju konacna verzija
\frontmatter

% Tekst diplomskog ...

% Diplomski rad treba poceti s uvodnim poglavljem  
\chapter*{Uvod}
\section[Problem grupiranja podataka][PGP]{Problem grupiranja podataka}
\section[Primjena][primjena]{Primjena}
\section[Pregled rada][pregled]{Pregled rada}

\chapter[Modeliranje problema grupiranja][modeliranje]{Modeliranje problema grupiranja}	
% ukoliko naslov nije jako dugacak dovoljno je samo \chapter{Naslov poglavlja} 

\section[Osnovni pojmovi][os-pojmovi]{Osnovni pojmovi}
Kako bi kasnije bilo jednostavnije objašnjavati strukture i same implementacije algoritama potrebno je problem grupiranja reprezentirati osnovnim pojmovima. U nastavku ćemo formalno definirati sve komponente od kojih se problem grupiranja sastoji.
\begin{defn}
\label{def:uzorak}
\textbf{Uzorak} je apstraktna struktura podataka koja reprezentira stvarne podatke s kojima raspolaže algoritam za klasteriranje.
\end{defn}
\begin{defn}
\label{def:svojstvo}
\textbf{Svojstvo} je vrijednost ili struktura koja predstavlja jednu značajku danog podatka unutar uzorka.
\end{defn}
\begin{defn}
\label{def:udaljenost}
\textbf{Udaljenost} između uzoraka definiramo kao funkciju 
$f: D  -> \R$, gdje je $D$ skup svojstava danih uzoraka
\end{defn}
\begin{defn}
\label{def:blizina}
Za uzorke kažemo da su \textbf{blizu} jedan drugome ako je njihova udaljenost manja od unaprijed zadane veličine
\end{defn}
\begin{defn}
\label{def:klaster}
\textbf{Klaster} je skup uzoraka koji su u prostoru podataka blizu. Ako su uzorci identični onda je njihova udaljenost uvijek $0$
\end{defn}
\begin{defn}
\label{def:hard}
\textbf{Jednistveno grupiranje} je postupak grupiranja kada svaki uzorak pripada jednom i samo jednom klasteru.
\end{defn}
\begin{defn}
\label{def:fuzzy}
\textbf{Nejasno ili nejedinstveno grupiranje} je postupak grupiranja gdje jedan uzorak može biti u više klastera.
\end{defn}


\section[Matematičko modeliranje problema][MMP]{Matematičko modeliranje problema}
Definicija grupiranja podataka nije jedinstvena. U literaturama se na različite načine pokušava opisati ovaj postupak. Neki od pokušaja opisne definicije su:

\begin{enumerate}
\item \textit{Grupiranje podataka je postupak otkrivanja homogenih\footnote{podaci koji se ne mogu smisleno separirati} grupa uzoraka unutar skupa svih danih uzoraka.}

\item \textit{Grupiranje podataka je postupak određivanja koji su uzorci slični te ih svrstati u isti klaster.}
\end{enumerate}
\section[Metode razvoja algoritama za grupiranje][metode-razvoja]{Metode razvoja algoritama za grupiranje}
\section[Upravljanje podacima][upravljanje-podacima]{Upravljanje podacima}

\chapter{Meta-heuristički pristup problemu}
\section[Prirodom inspirirani algoritmi][prirodni-algoritmi]{Prirodom inspirirani algoritmi}
\section{Reprezentacija podataka}
\section{Analiza rezultata}
\chapter{Poznati algoritmi i analiza}
\section{Alg 1}
\section{Alg 2}
\section{Alg 3}
\chapter{Tehnike za paralelizaciju algoritama}
\section[Osnovni pojmovi MPI tehnologije][mpi]{Osnovni pojmovi MPI tehnologije}
\section[Topologija][topologija]{Topologije}
\section[Prednosti paralelizacije i cijena komunikacije][pred-man-paralel]{Prednosti paralelizacije i cijena komunikacije}
\chapter{Konstrukcija paralelnih heurističkih algoritama za grupiranje}
\section{Algoritam 1}
\subsection{Opis}
\subsection{Analiza}
\section{Algoritam 2}
\subsection{Opis}
\subsection{Analiza}
\section{Algoritam 3}
\subsection{Opis}
\subsection{Analiza}
\chapter{Ostale moderne metode}
\section{Programiranje na grafičkim karticama}
\section{MapReduce metoda}
% Na kraju diplomkog rada stavlja se  bibliografija
% Najprije definiramo nacin prikazivanja bibliografije, u ovom slucaju verzija amsplain stila
\bibliographystyle{babamspl} % babamspl ili babplain

% U datoteku diplomski.bib se stavljaju bibliografske reference
% Bibliografske reference u bib formatu se mogu dobiti iz MathSciNet baze, Google Scholara, ArXiva, ...
\bibliography{diplomski}

\pagestyle{empty} % ne zelimo brojanje sljedecih stranica

% I na koncu idu sazeci na hrvatskom i engleskom

\begin{sazetak}
Ukratko ...
\end{sazetak}

\begin{summary}
In this ...
\end{summary}

% te zivotopis

\begin{cv}
Na slici \ref{def:fuzzy} se nalazi 3D graf neke funkcije. 

\begin{figure}[h!t]
\centering \includegraphics{surface3d.png}
\caption{Druga slika}
\label{fig:3d}
\end{figure}

kao i jedna vrlo komplicirana formula koja slijedi iz \eqref{eq:jed1}
\[ \sum_{i=1}^{\infty}A_{x_1}\times A_{{\alpha}_2}\oslash\iint_{\Omega}x^2\ddagger\limsup_{n\in\N}\frac{\alpha+\theta+\gamma}{n^{\omega}}\;\;\text{je u stvari}\;\;\biguplus_{r\in\Q}\overline{\Xi_i \mathop\Theta_{\substack{j\in\C \\ j\ni i\Q}} \Upsilon^{k^j} \underset{\ast}{\Psi} \hslash\vert_{\{\alpha\}}}.\]
\end{cv}

\end{document}